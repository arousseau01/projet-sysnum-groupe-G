\documentclass[a4paper]{article}
\usepackage[T1]{fontenc}
\usepackage[utf8]{inputenc}
\usepackage{lmodern}
\usepackage{amsmath,amssymb}
\usepackage[top=3cm,bottom=2cm,left=2cm,right=2cm]{geometry}
\usepackage{fancyhdr}
\usepackage{esvect,esint}
\usepackage{xcolor}
\usepackage{tikz,circuitikz}\usetikzlibrary{calc}

\parskip1em\parindent0pt\let\ds\displaystyle
\begin{document}
    Globalement, la chose qui m'a posé le plus de problème est l'interaction avec la machine : je suis capable d'aligner les lignes de code purement algorithmiques, mais toute interaction avec l'utilisateur notamment m'a été très difficile à mettre en oeuvre : je ne connaissais pas les modules, et ma compréhension de la documentation reste limitée par une compréhension limitée du fonctionnement interne de l'ordinateur.\\
    Dans la même veine, j'ai eu du mal à comprendre ce à quoi correspondait les types value et ty. Le fait de jongler avec les tableaux d'entiers, de booléens ou encore les chaînes de caractères m'a perturbé, en témoignent les nombreuses fonctions auxiliaires passant d'un de ces formats à l'autre.\\
    L'utilisation de ssh a aussi été un problème pour manipuler des fichiers linux, mais j'ai reçu de l'aide sur ce point donc j'ai été capable de m'en sortir.\\
    De manière générale, je ne manipule pas aisément un terminal.
\end{document}
