\documentclass[a4paper]{article}
\usepackage[T1]{fontenc}
\usepackage[utf8]{inputenc}
\usepackage{lmodern}
\usepackage{amsmath,amssymb}
\usepackage[top=3cm,bottom=2cm,left=2cm,right=2cm]{geometry}
\usepackage{fancyhdr}
\usepackage{esvect,esint}
\usepackage{xcolor}
\usepackage{tikz,circuitikz}\usetikzlibrary{calc}
\usepackage{longtable}

\title{ISA Groupe G}
\author{Pierre-Alexandre Bazin, Mathis Bouverot \& Arthur Rousseau}
\date{}
\parskip1em\parindent0pt\let\ds\displaystyle

\begin{document}
    \maketitle
    \section{Mémoire}
    La RAM et la ROM sont sur \(2^{16}\) bits, avec donc des adresses sur 16 bits. Les entiers manipulés seront toujours signés et sur 16 bits.
    
    \section{Registres}
    Il y a 8 registres nommés rax, rbx, rcx, rdx, rex, rfx, rhx, rz (registre nul).\\
    Ces registres sont sur 16 bits.\\
    Codage des registres :

    \begin{tabular}{|c|c|c|c|c|c|c|c|}
        \hline
        rz  & rax & rbx & rcx & rdx & rex & rfx & rhx \\
        \hline
        000 & 001 & 010 & 011 & 100 & 101 & 110 & 111 \\
        \hline  
    \end{tabular}

    \section{Instructions}
    Les instructions sont sur 32 bits. Leur schéma est le suivant :

    \begin{tabular}{|c|c|c|c|c|c|c|}
        \hline
        \texttt{imm} & \texttt{funct7} & \texttt{funct3} & \texttt{rs2} & \texttt{rs1} & \texttt{rd} & \texttt{opcode} \\
        \hline
    \end{tabular}

    avec \begin{itemize}
        \item \texttt{opcode} sur 3 bits
        \item \texttt{funct3} sur 3 bits
        \item \texttt{funct7} sur 1 bit 
        \item \texttt{rd}, \texttt{rs2}, \texttt{rs1} sur 3 bits (8 registres)
        \item \texttt{imm} sur 16 bits
    \end{itemize}

    Instructions possibles :

    \begin{longtable}{|l|l|c|c|c|l|}
        \hline
        Inst           & Nom                             & Opcode & funct3 & funct7 & Description                   \\
        \hline

        \texttt{add}   & ADD                             & 011    & 000    & 0      & \texttt{rd = rs1 + rs2}       \\
        \texttt{sub}   & SUB                             & 011    & 000    & 1      & \texttt{rd = rs1 - rs2}       \\
        \texttt{or}    & OR                              & 011    & 100    & 0      & \texttt{rd = rs1 or rs2}      \\
        \texttt{nand}  & NAND                            & 011    & 100    & 1      & \texttt{rd = rs1 nand rs2}    \\
        \texttt{xor}   & XOR                             & 011    & 001    & 0      & \texttt{rd = rs1 xor rs2}     \\
        \texttt{nxor}  & NXOR                            & 011    & 001    & 1      & \texttt{rd = rs1 nxor rs2}    \\
        \texttt{and}   & AND                             & 011    & 101    & 0      & \texttt{rd = (rs1 and rs2)}   \\
        \texttt{nor}   & NOR                             & 011    & 101    & 1      & \texttt{rd = rs1 nor rs2}     \\
        \texttt{sll}   & Shift left logical              & 011    & 011    & 0      & \texttt{rd = rs1 << rs2}      \\
        \texttt{srl}   & Shift right logical             & 011    & 010    & 0      & \texttt{rd = rs1 >> rs2}      \\
        \texttt{sra}   & Shift right Arith               & 011    & 010    & 1      & \texttt{rd = rs1 >> rs2}      \\
        \texttt{seq}   & Set equal                       & 011    & 111    & 1      & \texttt{rd = (rs1 = rs2)?1:0} \\
        \texttt{slt}   & Set less than                   & 011    & 110    & 1      & \texttt{rd = (rs1 < rs2)?1:0} \\
        \hline
        \texttt{addi}  & ADD (immediate)                 & 001    & 000    & 0      & \texttt{rd = rs1 + imm}       \\
        \texttt{subi}  & SUB (immediate)                 & 001    & 000    & 1      & \texttt{rd = rs1 - imm}       \\
        \texttt{ori}   & OR (immediate)                  & 001    & 100    & 0      & \texttt{rd = rs1 or imm}      \\
        \texttt{nandi} & NAND (immediate)                & 001    & 100    & 1      & \texttt{rd = rs1 nand imm}    \\
        \texttt{xori}  & XOR (immediate)                 & 001    & 001    & 0      & \texttt{rd = rs1 xor imm}     \\
        \texttt{nxori} & NXOR (immediate)                & 001    & 001    & 1      & \texttt{rd = rs1 nxor imm}    \\
        \texttt{andi}  & AND (immediate)                 & 001    & 101    & 0      & \texttt{rd = rs1 and imm}     \\
        \texttt{nori}  & NOR (immediate)                 & 001    & 101    & 1      & \texttt{rd = rs1 nor imm}     \\
        \texttt{slli}  & Shift left logical (immediate)  & 001    & 011    & 0      & \texttt{rd = rs1 << imm}      \\
        \texttt{srli}  & Shift right logical (immediate) & 001    & 010    & 0      & \texttt{rd = rs1 >> imm}      \\
        \texttt{srai}  & Shift right Arith (immediate)   & 001    & 010    & 1      & \texttt{rd = rs1 >> imm}      \\
        \texttt{seqi}  & Set equal (immediate)           & 001    & 111    & 1      & \texttt{rd = (rs1 = imm)?1:0} \\
        \texttt{slti}  & Set less than (immediate)       & 001    & 110    & 1      & \texttt{rd = (rs1 < imm)?1:0} \\
        \hline
        \texttt{lw}    & Load word                       & 000    & 000    & 0      & \texttt{rd=M[rs1]}            \\
        \texttt{sw}    & Store word                      & 010    & 000    & 0      & \texttt{M[rs1]=rs2}           \\
        \hline
        \texttt{beq}   & Branch ==                       & 110    & 100    & 1      & \texttt{if(rs1 == rs2) PC=imm}\\
        \texttt{bne}   & Branch !=                       & 110    & 101    & 1      & \texttt{if(rs1 != rs2) PC=imm}\\
        \texttt{ble}   & Branch \(\leqslant\)            & 110    & 110    & 1      & \texttt{if(rs1 <= rs2) PC=imm}\\
        \texttt{blt}   & Branch <                        & 110    & 010    & 1      & \texttt{if(rs1 < rs2) PC=imm} \\
        \texttt{bge}   & Branch \(\geqslant\)            & 110    & 011    & 1      & \texttt{if(rs1 >= rs2) PC=imm}\\
        \texttt{bgt}   & Branch >                        & 110    & 111    & 1      & \texttt{if(rs1 > rs2) PC=imm} \\
        \hline
        \texttt{jal}   & Jump and link                   & 111    & 000    & 0      &\texttt{rd=PC+4; PC=imm}         \\
        \texttt{jalr}  & Jump and link reg               & 101    & 000    & 0      &\texttt{rd=PC+4; PC=rs1+imm}     \\
        \texttt{jr}    & Jump reg                        & 100    & 000    & 0      &\texttt{PC=rs1+imm}            \\
        \texttt{jmp}   & Jump                            & 110    & 001    & 1      &\texttt{PC=imm}                \\
        \hline
    \end{longtable}

    On ajoute du sucre syntaxique:

    \begin{longtable}{|l|l|l|}
        \hline
        Inst          & Nom             & Description       \\
        \hline
        \texttt{mov}  & MOV             & \texttt{rd=rs1+0} \\
        \texttt{movi} & MOV (immediate) & \texttt{rd=imm+0} \\
        \hline
    \end{longtable}
\end{document}
