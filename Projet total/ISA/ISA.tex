\documentclass[a4paper]{article}
\usepackage[T1]{fontenc}
\usepackage[utf8]{inputenc}
\usepackage{lmodern}
\usepackage{amsmath,amssymb}
\usepackage[top=3cm,bottom=2cm,left=2cm,right=2cm]{geometry}
\usepackage{fancyhdr}
\usepackage{esvect,esint}
\usepackage{xcolor}
\usepackage{tikz,circuitikz}\usetikzlibrary{calc}

\parskip1em\parindent0pt\let\ds\displaystyle

\begin{document}
    8 registres : rax, rbx, rcx, rdx, rex, rfx, rhx, rz (registre 0).\\
    registres 16 bits.\\
    \begin{tabular}{|c|c|}
        \hline
        rz  & 000\\
        \hline
        rax & 001\\
        \hline
        rbx & 010\\
        \hline
        rcx & 011\\
        \hline
        rdx & 100\\
        \hline 
        rex & 101\\
        \hline 
        rfx & 110\\
        \hline
        rhx & 111\\
        \hline  
    \end{tabular}

    Instructions sur 32 bits.\\
    On réduit la taille de RISC-V:\begin{itemize}
        \item \texttt{opcode} sur 3 bits
        \item \texttt{funct3} sur 3 bits ()
        \item \texttt{funct7} sur 1 bit 
        \item \texttt{rd},\texttt{rs2},\texttt{rs1} sur 3 bits (8 registres)
        \item \texttt{imm} sur 16 bits
    \end{itemize}
    Arithmétique signée seulement : ADD, SUB, XOR, OR, AND, NAND, Shift (left logical, right logical, right arith).\\
    Load, store 32b.\\
    Branch, jump.

    \begin{tabular}{|c|c|c|c|c|c|c|}
        \hline
        \texttt{imm} & \texttt{funct7} & \texttt{funct3} & \texttt{rs2} & \texttt{rs1} & \texttt{rd} & \texttt{opcode} \\
        \hline
    \end{tabular}

    \begin{tabular}{|l|l|c|c|c|l|}
        \hline
        Inst & Nom                  & Opcode & funct3 & funct7 & Description                   \\
        \hline
        add  & ADD                  & 011    & 000    & 0      & \texttt{rd = rs1 + rs2}       \\
        sub  & SUB                  & 011    & 000    & 1      & \texttt{rd = rs1 - rs2}       \\
        or   & OR                   & 011    & 100    & 0      & \texttt{rd = rs1 or rs2}      \\
        xor  & XOR                  & 011    & 110    & 0      & \texttt{rd = rs1 xor rs2}     \\
        and  & AND                  & 011    & 101    & 0      & \texttt{rd = (rs1 and rs2)}   \\
        nand & NAND                 & 011    & 100    & 1      & \texttt{rd = rs1 nand rs2}    \\
        nor  & NOR                  & 011    & 101    & 1      & \texttt{rd = rs1 nor rs2}     \\
        nxor & NXOR                 & 011    & 110    & 1      & \texttt{rd = rs1 nxor rs2}    \\
        sll  & Shift left logical   & 011    & 011    & 0      & \texttt{rd = rs1 << rs2}      \\
        srl  & Shift right logical  & 011    & 010    & 0      & \texttt{rd = rs1 >> rs2}      \\
        sra  & Shift right Arith    & 011    & 010    & 1      & \texttt{rd = rs1 >> rs2}      \\
        slt  & Set less than        & 011    & 001    & 1      & \texttt{rd = (rs1 < rs2)?1:0} \\
        \hline
        lw   & Load word            & 000    & 000    & 0      & \texttt{rd=M[rs1]}            \\
        sw   & Store word           & 010    & 000    & 0      & \texttt{M[rs1]=rs2}           \\
        \hline
        beq  & Branch ==            & 110    & 000    & 0      & \texttt{if(rs1 == rs2) PC=imm}\\
        bne  & Branch !=            & 110    & 000    & 1      & \texttt{if(rs1 != rs2) PC=imm}\\
        ble  & Branch \(\leqslant\) & 110    & 001    & 0      & \texttt{if(rs1 <= rs2) PC=imm}\\
        blt  & Branch <             & 110    & 001    & 1      & \texttt{if(rs1 < rs2) PC=imm} \\
        bge  & Branch \(\geqslant\) & 110    & 011    & 0      & \texttt{if(rs1 >= rs2) PC=imm}\\
        bgt  & Branch >             & 110    & 011    & 1      & \texttt{if(rs1 > rs2) PC=imm} \\

        \hline
        jal  & Jump and link        & 101    & 000    & 0      &\texttt{rd=PC; PC=imm}         \\
        jalr & Jump and link reg    & 111    & 000    & 0      &\texttt{rd=PC; PC=rs1+imm}     \\
        \hline
    \end{tabular}
\end{document}
