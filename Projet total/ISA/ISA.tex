\documentclass[a4paper]{article}
\usepackage[T1]{fontenc}
\usepackage[utf8]{inputenc}
\usepackage{lmodern}
\usepackage{amsmath,amssymb}
\usepackage[top=3cm,bottom=2cm,left=2cm,right=2cm]{geometry}
\usepackage{fancyhdr}
\usepackage{esvect,esint}
\usepackage{xcolor}
\usepackage{tikz,circuitikz}\usetikzlibrary{calc}

\parskip1em\parindent0pt\let\ds\displaystyle

\begin{document}
    8 registres : rax, rbx, rcx, rdx, rex, rfx, rhx, rz (registre 0).\\
    registres 3 bits, stockent 32/64 bits.\\
    Instructions sur 32 bits.\\
    On réduit la taille de RISC-V:\begin{itemize}
        \item opcode sur 3 bits
        \item funct3 sur 3 bits ()
        \item funct7 sur 1 bit 
        \item rd,rs2,rs1 sur 3 bits (8 registres)
    \end{itemize}
    Arithmétique signée seulement : ADD, SUB, XOR, OR, AND, NAND, Shift (left logical, right logical, right arith).\\
    Load, store 32b.\\
    Branch, jump.

    \begin{tabular}{|c|c|c|c|c|c|c|}
        \hline
        imm & funct7 & funct3 & rs2 & rs1 & rd & opcode \\
        \hline
    \end{tabular}

    \begin{tabular}{|l|l|c|c|c|c|}
        \hline
        Inst & Nom                  & FMT & Opcode & funct3 & funct7 \\
        \hline
        add  & ADD                  & R   & 011    & 000    & 0      \\
        sub  & SUB                  & R   & 011    & 000    & 1      \\
        or   & OR                   & R   & 011    & 100    & 0      \\
        xor  & XOR                  & R   & 011    & 110    & 0      \\
        and  & AND                  & R   & 011    & 101    & 0      \\
        nand & NAND                 & R   & 011    & 100    & 1      \\
        nor  & NOR                  & R   & 011    & 101    & 1      \\
        nxor & NXOR                 & R   & 011    & 110    & 1      \\
        sll  & Shift left logical   & R   & 011    & 011    & 0      \\
        srl  & Shift right logical  & R   & 011    & 010    & 0      \\
        sra  & Shift right Arith    & R   & 011    & 010    & 1      \\
        slt  & Set less than        & R   & 011    & 001    & 1      \\
        \hline
        lw   & Load word            & I   & 000    & 000    &        \\
        sw   & Store word           & S   & 010    & 000    &        \\
        \hline
        beq  & Branch ==            & B   & 110    & 000    &        \\
        bne  & Branch !=            & B   & 110    & 001    &        \\
        blt  & Branch <             & B   & 110    & 100    &        \\
        bge  & Branch \(\leqslant\) & B   & 110    & 101    &        \\
        \hline
        jal  & Jump and link        & J   & 101    &        &        \\
        jalr & Jump and link reg    & I   & 111    & 000    &        \\
        \hline
    \end{tabular}
\end{document}
