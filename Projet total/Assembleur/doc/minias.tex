\documentclass[a4paper]{article}
\usepackage[T1]{fontenc}
\usepackage[utf8]{inputenc}
\usepackage{lmodern}
\usepackage{amsmath,amssymb}
\usepackage[top=3cm,bottom=2cm,left=2cm,right=2cm]{geometry}

\title{Assembleur Groupe G (minias)}
\author{Pierre-Alexandre Bazin, Mathis Bouverot \& Arthur Rousseau}
\date{}
\parskip1em\parindent0pt\let\ds\displaystyle

\begin{document}
    \maketitle
    \section{Registres}
    Il y a 7 registres minias nommés rax, rbx, rcx, rdx, rex, rfx, rhx.\\
    Ces registres sont sur 16 bits.\\
    Le registre ISA rz n'est pas accessible au programmeur.

    Pour accéder à l'adresse $x$ de la ram, en supposant $x$ dans un registre, 
    il faut écrire le nom du registre entre parenthèses.\\
    Exemple : \texttt{mov (\%rax) \%rbx}
    
    \section{Instructions}
    Les instructions ne sont pas exactement celles de l'ISA.\\
    Les instructions ISA addi, subi, etc. n'existent pas dans minias (utiliser add, sub, etc. à la place).\\ 
    Les instructions ISA lw, sw n'existent pas dans minias (utiliser mov à la place).\\
    L'instruction minias mov est traduite en différentes instructions ISA (lw, sw, addi, etc.) en fonction des arguments.

    Syntaxe des instructions :
    \begin{itemize}
        \item add, sub, or, nand, xor, nxor, and, nor, sll, srl, sra, seq, slt :\\
        \texttt{instr dest source1 source2}
        \begin{itemize}
            \item \texttt{instr} : nom de l'instruction
            \item \texttt{dest} : registre
            \item \texttt{source1} : registre
            \item \texttt{source2} : registre ou immediate
        \end{itemize}
        Exemple : \texttt{add \%rax \%rbx 42} 
        (fait \texttt{rax $\leftarrow$ rbx + 42}).
        \item beq, bne, ble, blt, bge, bgt :\\
        \texttt{instr arg1 arg2 label}
        \begin{itemize}
            \item \texttt{instr} : nom de l'instruction
            \item \texttt{arg1} : registre
            \item \texttt{arg2} : registre
            \item \texttt{label} : chaine de caractères 
        \end{itemize}
        Exemple : \texttt{beq \%rax \%rbx .L0} 
        \item mov :\\
        \texttt{mov dest source}
        \begin{itemize}
            \item \texttt{dest} : registre ou mémoire
            \item \texttt{source} : registre, mémoire ou immediate
        \end{itemize}
    \end{itemize}

    \section{Labels}
    Un label est une chaine de caractères comprenant des chiffres, 
    des underscores, des points et au moins une lettre (minuscule ou majuscule).\\
    Un label peut apparaitre avant une instruction suivi de deux points.\\
    Exemple :\\
    \texttt{.L1:\\beq \%rax \%rbx .L1}

\end{document}